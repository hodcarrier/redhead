\documentclass{article}
\usepackage{xeCJK}
\usepackage{fancyhdr}
\usepackage{xcolor}
\usepackage{draftwatermark}


%\setlength\hsize{156mm}
%\setlength\vsize{225mm}

\setlength\textwidth{156mm}
\setlength\textheight{225mm}

%\setlength\topskip{7mm}

   \setmainfont{方正仿宋简体}[]
\setCJKmainfont{方正仿宋简体}[]

% \CJKfontspec{方正小标宋简体}


\begin{document}

\fontsize{15.85}{15.85}
\setbox37=\hbox{缩进}
\parindent=\wd37
\baselineskip=10.446428571428571mm

\hoffset=-56pt
\voffset=1pt


公文用纸天头(上白边)为 37 mm±1 mm,
公文用纸订口(左白边)为 28mm±1mm,
版心尺寸为 156 mm×225 mm


10 公文的特定格式

10.1 信函格式

发文机关标志使用发文机关全称或者规范化简称, 居中排布, 上边缘至上页边为 30mm, 推荐
使用红色小标宋体字。 联合行文时, 使用主办机关标志。

发文机关标志下 4 mm 处印一条红色双线(上粗下细), 距下页边 20 mm 处印一条红色双线(上
细下粗), 线长均为 170 mm, 居中排布。

如需标注份号、 密级和保密期限、 紧急程度, 应当顶格居版心左边缘编排在第一条红色双线下,
按照份号、 密级和保密期限、 紧急程度的顺序自上而下分行排列, 第一个要素与该线的距离为 3 号
汉字高度的 7/8。

发文字号顶格居版心右边缘编排在第一条红色双线下, 与该线的距离为 3 号汉字高度的 7/8。
标题居中编排, 与其上最后一个要素相距二行。

第二条红色双线上一行如有文字, 与该线的距离为 3 号汉字高度的 7/8。
首页不显示页码。

版记不加印发机关和印发日期、 分隔线, 位于公文最后一面版心内最下方。

\the\textwidth

\the\vsize
\the\hsize



\pagestyle{fancy}





\fancyhead[L,R]{}
\fancyhead[C]{{\vskip -36pt\color{red}\noindent\CJKfontspec{方正小标宋简体}\fontsize{36}{32}惠州端口大婶科技有限公司}}
\fancyfoot[L]{}
\fancyfoot[C]{}
\fancyfoot[R]{\thepage}
\renewcommand{\headrulewidth}{0.4pt}
\renewcommand{\footrulewidth}{2pt}

%\renewcommand{\headrule}{\vbox to 0pt{\makebox[\headwidth]{\dotfill}\vss}}


\setlength\headwidth{170mm}

\newdimen\doublelineskip % 两横线间的距离
\setlength\doublelineskip{2pt}
\renewcommand\headrule{%

  {\color{red}%
  \vskip 4mm%
  \hrule height\headrulewidth width\headwidth%
  \vskip \doublelineskip%
  \hrule height 1pt width\headwidth}%
  }
  
\renewcommand\footrule{{%
  \color{red}%
  \vskip 4mm%
  \hrule height 1pt width \headwidth%
  \vskip \doublelineskip%
  \hrule height \headrulewidth width\headwidth%
  }}


\renewcommand{\headrulewidth}{1mm}
\renewcommand{\footrulewidth}{1mm}



\the\topskip
\vfill
《琵琶行》的字数刚好是一个版面,28字每行,总22行,
请观察下一页面效果。
\begin{center}
琵\quad 琶\quad 行

唐·白居易
\end{center}
\newpage
\noindent%
浔阳江头夜送客 枫叶荻花秋瑟瑟
主人下马客在船 举酒欲饮无管弦%1
醉不成欢惨将别 别时茫茫江浸月
%
忽闻水上琵琶声 主人忘归客不发%2
寻声暗问弹者谁 琵琶声停欲语迟
移船相近邀相见 添酒回灯重开宴%3
千呼万唤始出来 犹抱琵琶半遮面
转轴拨弦三两声 未成曲调先有情%4
弦弦掩抑声声思 似诉平生不得志
低眉信手续续弹 说尽心中无限事%5
轻拢慢捻抹复挑 初为霓裳后六幺
大弦嘈嘈如急雨 小弦切切如私语%6
嘈嘈切切错杂弹 大珠小珠落玉盘
间关莺语花底滑 幽咽泉流冰下难%7
冰泉冷涩弦凝绝 凝绝不通声暂歇
别有幽愁暗恨生 此时无声胜有声%8
银瓶乍破水浆迸 铁骑突出刀枪鸣
曲终收拨当心画 四弦一声如裂帛%9
东船西舫悄无言 唯见江心秋月白
%
沉吟放拨插弦中 整顿衣裳起敛容%10
自言本是京城女 家在虾蟆陵下住
十三学得琵琶成 名属教坊第一部%11
曲罢曾教善才服 妆成每被秋娘妒
五陵年少争缠头 一曲红绡不知数%12
钿头银篦击节碎 血色罗裙翻酒污
今年欢笑复明年 秋月春风等闲度%13
弟走从军阿姨死 暮去朝来颜色故
门前冷落鞍马稀 老大嫁作商人妇%14
商人重利轻别离 前月浮梁买茶去
去来江口守空船 绕船月明江水寒%15
夜深忽梦少年事 梦啼妆泪红阑干
%
我闻琵琶已叹息 又闻此语重唧唧%16
同是天涯沦落人 相逢何必曾相识
我从去年辞帝京 谪居卧病浔阳城%17
浔阳地僻无音乐 终岁不闻丝竹声
住近湓江地低湿 黄芦苦竹绕宅生%18
其间旦暮闻何物 杜鹃啼血猿哀鸣
春江花朝秋月夜 往往取酒还独倾%19
岂无山歌与村笛 呕哑嘲哳难为听
今夜闻君琵琶语 如听仙乐耳暂明%20
莫辞更坐弹一曲 为君翻作琵琶行
感我此言良久立 却坐促弦弦转急%21
凄凄不似向前声 满座重闻皆掩泣
座中泣下谁最多 江州司马青衫湿%22

% 这里应该刚好换页

\begin{center}
长\qquad\qquad 征

毛泽东

红军不怕远征难 万水千山只等闲
五岭逶迤腾细浪 乌蒙磅礴走泥丸
金沙水拍云崖暖 大渡桥横铁索寒
更喜岷山千里雪 三军过后尽开颜
  
\end{center}

\vfill
%\columnsep: 列间距
%\topmargin: 页眉到页边的距离
%\topskip: 页眉与正文的距离
%\textheight: 正文的高度
%\textwidth: 文本的宽度
%\oddsidemargin: 奇数页的左面页边距
%\evensidemargin : 偶数页的左面页边距
英文字母

The quick fox jumps over the lazy dog.

\end{document}
